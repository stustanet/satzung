%% $Project: StuStaNet-Satzung $
%% $ProjectVersion: 4.3 $
%% $ProjectDate: Tue, 12 Nov 2002 18:04:47 +0100 $
%% $Source: satzung.tex $
%% $Revision: 1.13 $

\documentclass[12pt]{article}

\usepackage{german}
\usepackage{a4wide}
\usepackage[utf8]{inputenc}

% Absatzeinrückung, -abstand und Zeilenabstand
\parindent0em
\parskip1.5ex plus0.5ex minus0.5ex

\newcommand{\Abschnitt}[1]{\section{#1}}
\newcommand{\UAbschnitt}[1]{\subsection{#1}}
\newcommand{\Satz}[2]{

\begin{samepage}
{\bf (#1)} #2
\end{samepage}
}


\newenvironment{Artikel}[2]{
\bigskip \centerline{{\bf Art. #1} [#2]}
\nopagebreak
}{
}

%\pagestyle{fancyplain}

\parindent0em
\parskip1.0ex plus0.1ex minus0.1ex

\title{\bf Satzung des Netzvereins StuStaNet e.V.}
\date{12. Dezember 2001}
\begin{document}
\maketitle

\Abschnitt{Grundlagen}

\begin{Artikel}{1}{Name und Sitz}

\Satz{1}{Der Verein führt den Namen {\it StuStaNet}.}

\Satz{2}{Er soll ins Vereinsregister eingetragen werden und erhält
daraufhin den Namenszusatz e.V.}

\Satz{3}{Sitz des Vereins ist München.}

\end{Artikel}

\begin{Artikel}{2}{Zweck}

\Satz{1}{Zweck des Verein ist die Studentenhilfe. Dazu errichtet und unterhält
er in der Studentenstadt Freimann ein Rechnernetz mit Internet-Anbindung.
Darüber hinaus will der Verein durch Informations- und Schulungsveranstaltungen
den Studenten Kompetenz im Umgang mit den Neuen Medien vermitteln.}

\Satz{2}{Um den Vereinszweck zu erreichen, arbeitet der Verein insbesondere mit
dem Studentenwerk München und dem Leibniz-Rechenzentrum der Bayerischen
Akademie der Wissenschaften zusammen.}

\end{Artikel}

\begin{Artikel}{3}{Gemeinnützigkeit}

Der Verein verfolgt ausschließlich und unmittelbar gemeinnützige Zwecke im
Sinne des Abschnitts {\it Steuerbegünstigte Zwecke} der Abgabenordnung.  Der
Verein ist selbstlos tätig; er verfolgt nicht in erster Linie
eigenwirtschaftliche Zwecke. Mittel des Vereins dürfen nur für die
satzungsmäßigen Zwecke verwendet werden. Die Mitglieder erhalten keine
Zuwendungen aus Mitteln des Vereins. Es darf keine Person durch Ausgaben, die
dem Zweck des Vereins fremd sind, oder durch unverhältnismäßig hohe Vergütungen
begünstigt werden.

\end{Artikel}


\begin{Artikel}{4}{Geschäftsjahr}

Geschäftsjahr des Vereins ist das Kalenderjahr. Das erste Rumpfgeschäftsjahr
endet am 31.12.1998.

\end{Artikel}

\begin{Artikel}{5}{Allgemeine Bestimmungen}

\Satz{1}{Ordnungsgemäß einberufene Gremien sind beschlußfähig, wenn mindestens
zwei stimmberechtigte Mitglieder anwesend sind. Sitzungen an Samstagen, Sonn-
und Feiertagen sind unzulässig, es sei denn, das Gremium ist vollzählig.
Insbesondere ist auf die Einhaltung einer angemessenen Frist zwischen
Einberufung und Sitzungstermin zu achten.}

\Satz{2}{Enthaltungen werden als ungültige Stimmen gewertet. Soweit diese
Satzung nichts anderes bestimmt, werden Beschlüsse mit der Mehrheit der
gültigen Stimmen gefaßt. Gewählt ist, wer mehr als die Hälfte der gültig
abgegebenen Stimmen erhält. Erreicht dies kein Kandidat, findet eine Stichwahl
zwischen den beiden bestplatzierten statt.}

\Satz{3}{Von jeder Sitzung wird ein Protokoll angefertigt und an den Vorstand
weitergeleitet. Beschlüsse werden vom Vorstand im Netz veröffentlicht.
Einspruch ist binnen zweier Monate nach der Veröffentlichung an den Vorstand zu
richten.}

\Satz{4}{Von dieser Satzung, allen Vereins- und Geschäftsordnungen, und den
Nutzungsbedingungen muß die aktuelle Fassung im Netz zugänglich sein.}

\Satz{5}{Im Eingangsbereich des Roten Hauses, Christoph-Probst-Str. 6, 80805
Mün\-chen, wird eine Anschlagtafel angebracht. Diese ist der offizielle
Vereinsanzeiger. Zusätzlich werden die Mitteilungen unter der URL
"`http://vereinsanzeiger.stustanet.de/"' veröffentlicht.  Hat ein Mitglied
keine Gelegenheit, den Vereinsanzeiger einzusehen, so kann er bis 1. April bzw.
1. Oktober für das jeweils folgende Semester beim Vorstand schriftlich
beantragen, auf dem Postweg zu den Versammlungen eingeladen zu werden.}

\end{Artikel}

\Abschnitt{Mitgliedschaft}

\begin{Artikel}{6}{Mitglieder}

\Satz{1}{Jeder Bewohner der Studentenstadt oder eines anderen vom Verein
betreuten Studentenwohnheims kann regelmäßig Mitglied werden. 
Die Aufnahme erfolgt auf schriftlichen Antrag an den Vorstand durch Ausstellung
einer Mitgliedskarte. Eine Ablehnung ist schriftlich zu begründen.  Art. 10
Abs. 3 ist anzuwenden.}

\end{Artikel}

\begin{Artikel}{7}{Rechte und Pflichten}

\Satz{1}{Jedes Mitglied hat eine Stimme auf der Mitgliedervollversammlung und
der Hausmitgliederversammlung seines Hauses.}

\Satz{2}{Vom Verein angebotene Netzdienste stehen grundsätzlich allen
Mitgliedern offen. Der Administratorenrat erläßt verbindliche
Nutzungsbedingungen.}

\Satz{3}{Anschriftenänderungen sind dem Vorstand unverzüglich anzuzeigen.}

\end{Artikel}

%\pagebreak

\begin{Artikel}{8}{Beiträge}

\Satz{1}{Es werden Semesterbeiträge erhoben. Über die Höhe beschließt die
Mitgliedervollversammlung. Semesterbeiträge sind unaufgefordert bis \mbox{1.
April} bzw. \mbox{1. Oktober} für das folgende Semester im voraus zu leisten.
Bei Beitragsrückstand erfolgt die Streichung aus der Mitgliederliste gemäß Art.
10.}

\Satz{2}{Es werden Aufnahmebeiträge erhoben. Über die Höhe befindet die
Mitgliedervollversammlung oder bis zur Höhe von 25 EUR der Vorstand.}

\end{Artikel}

\begin{Artikel}{9}{Ehrenmitglieder}

\Satz{1}{Personen, die sich für den Verein besonders eingesetzt haben, können
auf Antrag eines Hauses, des Vorstands oder des Administratorenrats von der
Mitgliedervollversammlung zu Ehrenmitgliedern ernannt werden.}

\Satz{2}{Die Ehrenmitgliedschaft ist auf fünf Jahre befristet. Der Vorstand
kann sie auf Antrag des Ehrenmitgliedes um jeweils fünf Jahre verlängern.}

\Satz{3}{Ehrenmitglieder sind von der Beitragspflicht befreit. Im übrigen
bleiben ihre Rechte und Pflichten als Mitglied unberührt.}

\end{Artikel}

\begin{Artikel}{10}{Ende der Mitgliedschaft}

\Satz{1}{Die Mitgliedschaft ist grundsätzlich auf 4 Jahre befristet.
Sie endet jedoch immer erst zu dem Zeitpunkt, an dem der Vorstand das
Mitglied aus der Mitgliederliste streicht.
Die Mitgliedschaft kann vom Vorstand auf Antrag um jeweils bis zu 4 Jahre 
verlängert werden.
Bekleidet ein Mitglied ein Amt und würde seine Mitgliedschaft während seiner
Amtszeit enden, so verlängert sie sich automatisch um ein weiteres Jahr.}

\Satz{2}{Die Mitgliedschaft endet desweiteren:
\begin{itemize}
\item mit dem Tod des Mitglieds;
\item durch schriftliche Austrittserklärung, die einen Monat im voraus an
ein Vorstandsmitglied zu richten ist. 
\item durch Ausschluß aus dem Verein;
\item durch Streichung aus der Mitgliederliste.
\end{itemize}
Bereits geleistete Semesterbeiträge werden anteilig zurückerstattet, wenn
dieser Anteil 25 EUR übersteigt.}

\Satz{3}{Ein Mitglied, das in erheblichem Maß gegen die Vereinsinteressen
verstoßen hat, kann durch Beschluß des Vorstands aus dem Verein ausgeschlossen
werden. Vor dem Ausschluß ist das betroffene Mitglied schriftlich zu hören. Die
Entscheidung über den Ausschluß ist schriftlich zu begründen und dem Mitglied
mit Einschreiben gegen Rückschein an seine letztbekannte Anschrift zuzustellen.
Das Mitglied kann innerhalb einer Frist von einem Monat ab Zugang schriftlich
Berufung beim Vorstand einlegen. Über die Berufung entscheidet die
Mitgliedervollversammlung. Macht das Mitglied vom Recht der Berufung innerhalb
der Frist keinen Gebrauch, unterwirft es sich dem Ausschließungsbeschluß.}

\Satz{4}{Die Streichung des Mitglieds aus der Mitgliederliste erfolgt durch den
Vorstand, wenn das Mitglied einen Monat mit dem Semesterbeitrag in Verzug ist
und diesen Betrag auch nach schriftlicher Mahnung durch den Vorstand nicht
innerhalb von zwei Wochen von der Absendung der Mahnung an die letztbekannte
Anschrift des Mitglieds voll entrichtet. In der Mahnung muß auf die
bevorstehende Streichung der Mitgliedschaft hingewiesen werden.}

\Satz{5}{Die Streichung erfolgt regelmäßig ebenfalls, wenn das Mitglied an der
dem Verein zuletzt bekannt gegebenen Postanschrift keinen Wohnsitz mehr hat.}

\Satz{6}{Die Streichung erfolgt außerdem regelmäßig, wenn die Mitgliedschaft
abgelaufen ist und keine Verlängerung erfolgt ist.}

\end{Artikel}

\Abschnitt{Vermögen}

\begin{Artikel}{11}{Geldvermögen}

\Satz{1}{Die Geldmittel des Vereins werden vom Schatzmeister verwaltet.  Die
Buch\-führung hat sich gegebenenfalls an die Vorgaben des Finanzamtes zu
richten.}

\Satz{2}{Der Schatzmeister erstellt einen Haushaltsplan und einen
Liquiditätsplan.}

\end{Artikel}

\begin{Artikel}{12}{Sachvermögen}

\Satz{1}{Die Sachmittel des Vereins werden von den Hausadministratoren, von den
Sonderadministratoren, vom Vorsitzenden des Administratorenrates oder vom
Vorstand verwaltet, abhängig davon, ob mit ihnen Aufgaben der Häuser, des
Sonderbereichs, des Administratorenrates oder des Vorstandes bestritten werden.}

\end{Artikel}

\begin{Artikel}{13}{Gewalt über die technische Infrastruktur}

\Satz{1}{Das Rechnernetz ist bis zu den Endgeräten durch Vereinseigentum
aufzubauen. Seine Funktion darf nur von vereinseigener Hardware abhängen. 
Hardware des Studentenwerks, von beauftragten Firmen, des
Leibniz-Rechenzentrums oder anderer öffentlicher Einrichtungen sind davon
ausgenommen.}

\Satz{2}{Weitergehende Rechte auf einem vereinseigenen Rechner oder anderen
technischen Einrichtungen werden regelmäßig nur Administratoren, den
Mitgliedern des Vorstandes sowie Beauftragten des Studentenwerks eingeräumt.
Über begründete Ausnahmen beschließt der Administratorenrat im Einvernehmen mit
dem Vorstand.}

\Satz{3}{Die technischen Vorstände führen eine Liste, wer auf welchem der
vereinseigenen Rechner oder technischen Einrichtungen weitergehende Rechte
besitzt.}

\end{Artikel}

\Abschnitt{Organe}

\begin{Artikel}{14}{Organe des Vereins}

\Satz{1}{Organe des Vereins sind  
\begin{itemize}
\item die Mitgliedervollversammlung
\item die Häuser
\item der Vorstand
\item der Administratorenrat.
\end{itemize}}

\Satz{2}{Die Vertretung nach außen im Sinne des \S 26 BGB üben immer zwei
Vorstandsmitglieder gemeinsam aus. Mit Wirkung gegen Dritte wird allen Organen
untersagt, ohne Zustimmung der Mitgliedervollversammlung Kredite oder andere
Verpflichtungen mit einer Laufzeit von mehr als zwei Jahren aufzunehmen.}

\end{Artikel}

\UAbschnitt{Mitgliedervollversammlung}

\begin{Artikel}{15}{Aufgaben}

\Satz{1}{Die Mitgliedervollversammlung kümmert sich um alle Fragen mit
zentraler Bedeutung für den Gesamtverein.}

\Satz{2}{Sie hat insbesondere folgende Aufgaben:
\begin{itemize}
\item Genehmigung des Haushaltsplans für das folgende Geschäftsjahr
\item Entgegennahme des Rechenschaftsberichts des Vorstands und dessen 
Entlastung
\item Wahl des Vorstands, der Sonderadministratoren und des 
Kassenprüfungsausschusses
\item Festsetzung der Höhe des Mitgliedsbeitrags und der Aufnahmegebühren
\item Beschlußfassung über Satzungsänderungen und Vereinsauflösung
\item Beschlußfassung über die Berufung eines Mitglieds gegen einen 
Ausschluß durch den Vorstand
\item Die Einteilung der Studentenstadt und anderer vom Verein betreuten
Studentenwohnheime in Häuser
\item Sie entscheidet, welche Studentenwohnheime neben der Studentenstadt
vom Verein betreut werden.
\end{itemize}}

\Satz{3}{Sie kann eine Rahmengeschäftsordnung für alle Vereinsgremien, eine
Beitragsordnung oder eine Haushaltsordnung erlassen.}

\Satz{4}{Die Mitgliedervollversammlung kann grundsätzlich festlegen, dass
bestimmte Netzdienste angeboten werden sollen oder nicht angeboten werden
sollen. Möchte die Mitgliedervollversammlung bestimmte Netzdienste anbieten, so
ist die technische Infrastruktur dafür bevorzugt einzusetzen. Die Mitglieder
des Administratorenrates sind jedoch nicht verpflichtet, diese Netzdienste
selbst zu realisieren.}

\Satz{5}{Über die Beschlüsse der Mitgliedervollversammlung ist ein Protokoll
aufzunehmen, vom Versammlungsleiter zu unterzeichnen und vom Vorstand zu
ver\-öffent\-lichen.}

\end{Artikel}

\begin{Artikel}{16}{Einberufung}

\Satz{1}{Die ordentlichen Mitgliedervollversammlungen finden zweimal jährlich,
und zwar in den Zeiträumen 15. Mai - 15. Juni und 15. November - 15. Dezember, statt.}

\Satz{2}{Die Mitglieder werden mindestens zwei Wochen vor einer
Mitgliedervollversammlung vom Vorstand durch Aushang im Vereinsanzeiger
eingeladen, wobei der Tag der Ver\-öffent\-lichung und der Tag der Versammlung
nicht mitgezählt werden.}

\Satz{3}{Mit der Einladung der Mitgliedervollversammlung ist die vom Vorstand
festgesetzte Tagesordnung mitzuteilen.}

\Satz{4}{Der Vorstand hat unverzüglich  eine außerordentliche
Mitgliedervollversammlung einzuberufen, wenn das Vereinsinteresse es erfordert
oder wenn 15 v.H. der Mitglieder oder der Administratorenrat dies schriftlich
unter Angabe des Zwecks und der Gründe fordern.}

\end{Artikel}

\UAbschnitt{Häuser} 

\begin{Artikel}{17}{Gliederung in Häuser}

\Satz{1}{Der Verein ist in Häuser untergliedert. Die Häuser regeln ihre Belange
selbst. Der Vorstand und der Administratorenrat können den Häusern Weisungen
erteilen.}

\Satz{2}{Jedem Haus wird ein Netzbereich als Hausnetz zugeordnet.}

\Satz{3}{Die Mitglieder können die Zuordnung zu den Häusern nicht wählen. 
Ausschlaggebend ist der Standort ihres Anschlusses. Im Zweifel entscheidet der
Vorstand.}

\Satz{4}{Mitglieder, die nicht in einem vom Verein betreuten Studentenwohnheim
wohnen, gehören keinem Haus an.}

\end{Artikel}

\begin{Artikel}{18}{Zuständigkeit}

\Satz{1}{Die Häuser organisieren das Vereinsleben auf Hausebene.}

\Satz{2}{Sie verwalten ihr jeweiliges Hausnetz in eigener Regie. Die Vorgaben
des Administratorenrates werden dabei von ihnen beachtet.}

\end{Artikel}

\begin{Artikel}{19}{Hausadministratoren}

\Satz{1}{Jedes Haus hat einen Hausadministrator. Pro angefangene 20 Mitgliedern
darf ein weiterer gewählt werden. Ausschlaggebend ist die Mitgliederzahl am
Tag der Ladung zur jeweiligen Hausmitgliederversammlung.}

\Satz{2}{Die Hausadministratoren leiten ihre Häuser, verwalten die Hausnetze
und sind Mitglied im Administratorenrat.}

\Satz{3}{Die Amtsperioden der Hausadministratoren laufen vom \mbox{15. Mai} bis
\mbox{14. November} und vom \mbox{15. November} bis \mbox{14. Mai}.}

\Satz{4}{Die Wahl der Hausadministratoren ist dem Vorstand anzuzeigen.  Dieser
führt eine Liste der derzeitig tätigen Administratoren. Mit der Aufnahme in
diese Liste ist der Gewählte zum Administrator bestellt.}

\Satz{5}{Wählt ein Haus nicht mindestens einen Hausadministrator, oder sind
alle Hausadministratorenstellen eines Hauses verwaist, so setzt der
Administratorenrat einen Hausadministrator ein.}

\end{Artikel}

\begin{Artikel}{20}{Hausmitgliederversammlung}

\Satz{1}{Die Hausmitgliederversammlung regelt alle Belange ihres Hauses.}

\Satz{2}{Sie wählt die Hausadministratoren, überwacht deren Amtsführung und
beschließt über ihre Entlastung.}

\Satz{3}{Die Hausadministratoren legen ihr Fragen von besonderem Gewicht zur
Entscheidung vor.}

\Satz{4}{Den Vorsitz führt in der Regel einer der Hausadministratoren des
Hauses.}

\end{Artikel}

\begin{Artikel}{21}{Einberufung}

\Satz{1}{Die ordentlichen Hausmitgliederversammlungen finden zweimal jährlich,
und zwar in den Zeit\-räumen 15. April - 14. Mai und 15. Oktober - 14. November, statt. Sie
wird in der Regel von einem der Hausadministratoren des Hauses, im Ausnahmefall
vom Vorstand einberufen.}

\Satz{2}{Die Mitglieder werden mindestens zwei Wochen vor einer
Hausmitgliederversammlung von einem der Hausadministratoren durch Aushang im
Vereinsanzeiger eingeladen. Artikel 16, Abs. 2 und 3 gelten entsprechend.}

\Satz{3}{Die Hausadministratoren können eine außerordentliche
Hausmitgliederversammlung einberufen. Sie müssen das innerhalb von drei Wochen
tun, wenn 15 v.H. der Hausmitglieder dies fordern.}

\end{Artikel}

\UAbschnitt{Vorstand}

\begin{Artikel}{22}{Zusammmensetzung und Amtszeit}

\Satz{1}{Der Vorstand setzt sich aus dem Vorsitzenden, dem Schatzmeister, sowie
dem ersten und zweiten technischen Vorstand zusammen. Ist das Amt des dritten
technischen Vorstands besetzt, so gehört auch er dem Vorstand an.}

\Satz{2}{Der Vorstand wird auf ein Jahr gewählt. Die Wahl erfolgt geheim und
für jedes Vorstandsamt einzeln. Die Amtsperiode des Vorstandes
läuft vom 1. Juli bis 30. Juni. Er bleibt so lange im Amt, bis ein neuer
Vorstand sein Amt angetreten hat. Scheidet ein Mitglied während der Amtsperiode
aus, wählt der Administratorenrat für die Zeit bis zur nächsten
Mitgliedervollversammlung ein Ersatzmitglied.}

\Satz{3}{Der Vorstand haftet nicht für leichte Fahrlässigkeit.}

\end{Artikel}

\begin{Artikel}{23}{Geschäftsverteilung}

\Satz{1}{Der Vorsitzende kümmert sich um alle wichtigen Belange des Vereins. Er
koordiniert die Kontakte nach außen und beruft bei Bedarf Vorstandssitzungen
ein. Er führt die Administratorenliste und sammelt und publiziert die
Protokolle aller beschlußfassenden Gremien. Ferner sorgt er für die
Verfügbarkeit aller gültigen Regelwerke des Vereins in der jeweils aktuellen
Fassung.}

\Satz{2}{Der Schatzmeister verwaltet alles Geldvermögen. Er ist verantwortlich
für eine ordnungsgemäße Buchführung. Er führt die Mitgliederliste. Er vertritt
den Vorsitzenden, falls dieser verhindert ist. Der Schatzmeister kann nicht
gleichzeitig Administrator sein.}

\Satz{3}{Der erste technische Vorstand führt den Vorsitz im Administratorenrat.
Er muß mindestens 6 Monate lang Administrator gewesen sein oder mit
satzungsändernder Mehrheit gewählt werden.}

\Satz{4}{Der zweite technische Vorstand unterstützt und vertritt den ersten
technischen Vorstand in all seinen Aufgaben und ist erster stellvertretender
Vorsitzender des Administratorenrates.}

\Satz{5}{Der dritte technische Vorstand unterstützt ersten und zweiten
technischen Vorstand. Er vertritt diese in all ihren Aufgaben und ist zweiter
stellvertretender Vorsitzender des Administratorenrates.
Das Amt des dritten technischen Vorstands kann unbesetzt bleiben.}

\end{Artikel}

\begin{Artikel}{24}{Aufgaben}

\Satz{1}{Die Vorstände sind die kaufmännische und organisatorische 
Führung des Vereins. Sie haben auch im Falle ihrer Abwesenheit für die 
Abwicklung notwendiger Vorgänge Sorge zu tragen.}

\Satz{2}{Die Vorstände sollen die Entwicklung des Vereins fördern und die
finanzielle Stabilität sicherstellen.}

\Satz{3}{Ausgaben sind vom Vorstand zu bewilligen. Bei Ausgaben über 3.000 EUR
müssen mindestens vier Vorstandsmitglieder zustimmen, darunter der
Schatzmeister oder der Vorsitzende.}

\Satz{4}{Die technischen Vorstände sollen jederzeit auf dem aktuellen Stand der
technischen Infrastruktur sein.}

\end{Artikel}

\UAbschnitt{Administratorenrat}

\begin{Artikel}{25}{Zusammensetzung}

\Satz{1}{Mitglieder des Administratorenrates sind die Haus- und
Sonderadministratoren und die technischen Vorstände.}

\Satz{2}{Jedes Ratsmitglied hat eine Stimme. Vertretung ist unzulässig.}

\end{Artikel}

\begin{Artikel}{26}{Aufgaben}

\Satz{1}{Der Administratorenrat kümmert sich um alle technischen Belange.}

\Satz{2}{Er legt verbindliche technische Standards fest und verwaltet die
technische Infrastruktur des Vereins. Er entscheidet über ihren Ausbau, Umbau
und ihre Verwendung. Die dazu benötigten Mittel bewilligt der Vorstand
wohlwollend. Sollte der Vorstand die Bewilligung verweigern, so hat er dies
binnen zweier Wochen im Vereinsanzeiger umfassend zu begründen.}

\Satz{3}{Alle Administratoren stellen sicher, daß schnell auf alle für den
normalen Netzbetrieb notwendigen Komponenten zugegriffen werden kann.}

\end{Artikel}

\begin{Artikel}{27}{Vorsitz und Einberufung}

\Satz{1}{Der erste technische Vorstand ist Vorsitzender des
Administratorenrats. Der zweite technische Vorstand ist erster
stellvertretender Administratorenratsvorsitzender, der dritte technische
Vorstand ist zweiter stellvertretender Administratorenratsvorsitzender.}

\Satz{2}{Der Administratorenrat wird in der Regel vom ersten technischen
Vorstand einberufen. Bei Bedarf kann er von jedem Vorstandsmitglied
einberufen werden.}

\Satz{3}{Der Administratorenrat muß auf Verlangen von 4 seiner Mitglieder
innerhalb von 2 Wochen einberufen werden.}

\end{Artikel}

\begin{Artikel}{28}{Sonderadministratoren}

\Satz{1}{Die Amtsperioden der Sonderadministratoren laufen vom 1. Juli bis 31.
Dezember und vom 1. Januar bis 30. Juni.}

\Satz{2}{Die Sonderadministratoren werden von der Mitgliedervollversammlung
ge\-wählt und durch Eintragung in die Administratorenliste durch den Vorstand
bestellt.}

\end{Artikel}


\Abschnitt{Wenn alle Stricke reißen}

\begin{Artikel}{29}{Satzungsänderungen}

\Satz{1}{Satzungsänderungen beschließt die Mitgliedervollversammlung mit
Zweidrittelmehrheit auf gemeinsamen Antrag von Vorstand und Administratorenrat
oder von 15 v.H. der Mitglieder.}

\Satz{2}{Die Änderung ist im Vereinsregister einzutragen.}

\end{Artikel}

\begin{Artikel}{30}{Auflösung}

\Satz{1}{Die Auflösung bestimmt sich nach den Regeln der Satzungsänderung.}

\Satz{2}{Bei Auflösung des Vereins fällt das Vermögen an das Studentenwerk
München, Anstalt des öffentlichen Rechts, mit der Maßgabe es unmittelbar und
ausschließlich im Sinn des Vereinszwecks zu verwenden.}

\end{Artikel}

\begin{Artikel}{31}{Vollzug bei Satzungsänderungen}

Die Satzungsänderung ist durch den Vorstand zur Eintragung in das
Vereinsregister anzumelden. Dem Vorsitzenden ist das Recht übertragen, etwaige
Satzungsänderungen, die das Registergericht für die Eintragung einer
Satzungsänderung verlangen sollte, vorzunehmen.

\end{Artikel}

\end{document}


