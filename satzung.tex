%% $Project: StuStaNet-Satzung $
%% $ProjectVersion: 1.3 $
%% $ProjectDate: Fri, 25 Jun 1999 18:26:36 +0200 $
%% $Source: satzung.tex $
%% $Revision: 1.8 $

\documentclass[12pt]{article}

\usepackage{german}
\usepackage{a4wide}
\usepackage{umlaut}

% Absatzeinr�ckung, -abstand und Zeilenabstand
\parindent0em
\parskip1.5ex plus0.5ex minus0.5ex

\newcommand{\Abschnitt}[1]{\section{#1}}
\newcommand{\UAbschnitt}[1]{\subsection{#1}}
\newcommand{\Satz}[2]{

\begin{samepage}
{\bf (#1)} #2
\end{samepage}
}


\newenvironment{Artikel}[2]{
\bigskip \centerline{{\bf Art. #1} [#2]}
\nopagebreak
}{
}

%\pagestyle{fancyplain}

\parindent0em
\parskip1.0ex plus0.1ex minus0.1ex

\title{\bf Satzung des Netzvereins StuStaNet e.V.}
\date{14. Juli 1998}
\begin{document}
\maketitle

\Abschnitt{Grundlagen}

\begin{Artikel}{1}{Name und Sitz}

\Satz{1}{Der Verein f"uhrt den Namen {\it StuStaNet}.}

\Satz{2}{Er soll ins Vereinsregister eingetragen werden und erh"alt
daraufhin den Namenszusatz e.V.}

\Satz{3}{Sitz des Vereins ist M"unchen.}

\end{Artikel}

\begin{Artikel}{2}{Zweck}

\Satz{1}{Zweck des Verein ist die Studentenhilfe. Dazu errichtet und 
unterh"alt er in der Studentenstadt Freimann ein Rechnernetz mit 
Internet-Anbindung. Dar"uber hinaus will der Verein durch 
Informations- und Schulungsveranstaltungen den Studenten Kompetenz 
im Umgang mit den Neuen Medien vermitteln.}

\Satz{2}{Um den Vereinszweck zu erreichen, arbeitet der Verein insbesondere
mit dem Studentenwerk M"unchen und dem Leibniz-Rechenzentrum der Bayerischen 
Akademie der Wissenschaften zusammen.}

\end{Artikel}

\begin{Artikel}{3}{Gemeinn"utzigkeit}

Der Verein verfolgt aussschlie"slich und unmittelbar gemeinn"utzige Zwecke 
im Sinne des Abschnitts {\it Steuerbeg"unstigte Zwecke} der Abgabenordnung.
Der Verein ist selbstlos t"atig; er verfolgt nicht in erster Linie 
eigenwirtschaftliche Zwecke. Mittel des Vereins d"urfen nur f"ur die 
satzungsm"a"sigen Zwecke verwendet werden. Die Mitglieder erhalten keine 
Zuwendungen aus Mitteln des Vereins. Es darf keine Person durch Ausgaben,
die dem Zweck des Vereins fremd sind, oder durch unverh"altnism"a"sig hohe 
Verg"utungen beg"unstigt werden. 

\end{Artikel}


\begin{Artikel}{4}{Gesch"aftsjahr}

Gesch"aftsjahr des Vereins ist das Kalenderjahr. Das erste Rumpfgesch"aftsjahr
endet am 31.12.98.

\end{Artikel}

\begin{Artikel}{5}{Allgemeine Bestimmungen}

\Satz{1}{Ordnungsgem"a"s einberufene Gremien sind beschlu"sf"ahig, wenn
mindestens zwei stimmberechtigte Mitglieder anwesend sind. Sitzungen an 
Samstagen, Sonn- und Feiertagen sind unzul"assig, es sei denn, das Gremium 
ist vollz"ahlig. Insbesondere ist auf die Einhaltung einer angemessenen
Frist zwischen Einberufung und Sitzungstermin zu achten.}

\Satz{2}{Gew"ahlt ist, wer mehr als die H"alfte der g"ultig abgegebenen
Stimmen erh"alt. Erreicht dies kein Kandidat, findet eine Stichwahl zwischen
den beiden bestplazierten statt.}

\Satz{3}{Von jeder Sitzung wird ein Protokoll angefertigt und an den 
Vorstand weitergeleitet. Beschl"usse werden vom Vorstand im Netz
ver�ffentlicht. Einspruch ist binnen zweier Monate nach der Ver"offentlichung
an den Vorstand zu richten.}

\Satz{4}{Von dieser Satzung, allen Vereins- und Gesch"aftsordnungen, 
und den Nutzungsbedingungen mu"s die aktuelle Fassung im Netz zug"anglich sein.}

\Satz{5}{Im Eingangsbereich des Roten Hauses, Christoph-Probst-Str. 6,
80805 M"un\-chen, wird eine Anschlagtafel angebracht. Diese ist der 
offizielle Vereinsanzeiger. Hat ein Mitglied keine Gelegenheit, den 
Vereinsanzeiger einzusehen, so kann er bis 1. April bzw. 1. Oktober f"ur das 
jeweils folgende Semester beim Vorstand schriftlich beantragen, auf dem 
Postweg zu den Versammlungen eingeladen zu werden.}

\end{Artikel}

\Abschnitt{Mitgliedschaft}

\begin{Artikel}{6}{Mitglieder}

\Satz{1}{Jeder Bewohner der Studentenstadt kann regelm"a"sig Mitglied werden. 
Die Aufnahme erfolgt auf schriftlichen Antrag an den Vorstand durch 
Ausstellung einer Mitgliedskarte. Eine Ablehnung ist schriftlich zu begr�nden.
Art. 10 Abs. 3 ist anzuwenden.}

\Satz{2}{Es kann eine Aufnahmegeb"uhr erhoben werden. "Uber die H"ohe
entscheidet die Mitgliedervollversammlung.}

\end{Artikel}

\begin{Artikel}{7}{Rechte und Pflichten}

\Satz{1}{Jedes Mitglied hat eine Stimme auf der Mitgliedervollversammlung 
und der Hausmitgliederversammlung seines Hauses.}

\Satz{2}{Es erhalten grunds"atzlich nur Mitglieder Netzzugang. Der 
Administratorenrat erl"a"st verbindliche Nutzungsbedingungen.}

\Satz{3}{Anschriften"anderungen sind dem Vorstand unverz"uglich anzuzeigen.}

\end{Artikel}

%\pagebreak

\begin{Artikel}{8}{Beitr"age}

\Satz{1}{Es werden Semesterbeitr"age erhoben. "Uber die H"ohe beschlie"st die 
Mitgliedervollversammlung. Semesterbeitr"age sind 
unaufgefordert bis \mbox{1. April} bzw. \mbox{1. Oktober} f"ur das folgende 
Semester im voraus zu leisten. Bei Beitragsr"uckstand erfolgt die Streichung 
aus der Mitgliederliste gem"a"s Art. 10.}

\Satz{2}{Die ordentliche Mitgliedervollversammlung kann eine 
Sonderumlage zur Finanzierung wichtiger Anschaffungen beschlie"sen. 
Die H"ohe ist auf die H"alfte eines Semesterbeitrags beschr"ankt.}

\end{Artikel}

\begin{Artikel}{9}{Ehrenmitglieder}

\Satz{1}{Personen, die sich f"ur den Verein besonders eingesetzt haben,
k"onnen auf Antrag eines Hauses, des Vorstands oder des Administratorenrats
von der Mitgliedervollversammlung zu Ehrenmitgliedern ernannt werden.}

\Satz{2}{Die Ehrenmitgliedschaft ist auf f"unf Jahre befristet. Der
Vorstand kann sie auf Antrag des Ehrenmitgliedes um jeweils f"unf Jahre
verl"angern.}

\Satz{3}{Ehrenmitglieder sind von der Beitragspflicht befreit. 
Sie besitzen kein passives Wahlrecht.}

\end{Artikel}

\begin{Artikel}{10}{Ende der Mitgliedschaft}

\Satz{1}{Die Mitgliedschaft ist grunds"atzlich auf acht Semester befristet.
Sie kann vom Vorstand auf Antrag um jeweils zwei Semester verl"angert werden.}

\Satz{2}{Die Mitgliedschaft endet desweiteren:
\begin{itemize}
\item mit dem Tod des Mitglieds;
\item durch schriftliche Austrittserkl"arung, die einen Monat im voraus an
ein Vorstandsmitglied zu richten ist. 
\item durch Ausschlu"s aus dem Verein;
\item durch Streichung aus der Mitgliederliste.
\end{itemize}
Bereits geleistete Semesterbeitr"age werden anteilig zur"uckerstattet.}

\Satz{3}{Ein Mitglied, das in erheblichem Ma"s gegen die Vereinsinteressen
versto"sen hat, kann durch Beschlu"s des Vorstands aus dem Verein 
ausgeschlossen werden. Vor dem Ausschlu"s ist das betroffene Mitglied 
schriftlich zu h"oren. Die Entscheidung "uber den Aussschlu"s ist schriftlich 
zu begr"unden und dem Mitglied mit Einschreiben 
gegen R"uckschein an seine letztbekannte Anschrift zuzustellen. Das Mitglied
kann innerhalb einer Frist von einem Monat ab Zugang schriftlich Berufung beim 
Vorstand einlegen. "Uber die Berufung entscheidet die 
Mitgliedervollversammlung. Macht das Mitglied vom Recht der Berufung 
innerhalb der Frist keinen Gebrauch, unterwirft es sich dem 
Ausschlie"sungsbeschlu"s.}

\Satz{4}{Die Streichung des Mitglieds aus der Mitgliederliste erfolgt durch
den Vorstand, wenn das Mitglied einen Monat mit dem Semesterbeitrag in Verzug 
ist und diesen Betrag auch nach schriftlicher Mahnung durch den Vorstand nicht
innerhalb von zwei Wochen von der Absendung der Mahnung an die letztbekannte 
Anschrift des Mitglieds voll entrichtet. In der Mahnung mu"s auf die 
bevorstehende Streichung der Mitgliedschaft hingewiesen werden.}

\end{Artikel}

\Abschnitt{Verm"ogen}

\begin{Artikel}{11}{Geldverm"ogen}

\Satz{1}{Die Geldmittel des Vereins werden vom Schatzmeister verwaltet. 
Die Buch\-f"uhrung hat sich gegebenenfalls an die Vorgaben des Finanzamtes zu 
richten.}

\Satz{2}{Die Schatzmeister erstellt einen Haushaltsplan und einen
Liquidit�tsplan.}

\end{Artikel}

\begin{Artikel}{12}{Sachverm"ogen}

\Satz{1}{Die Sachmittel des Vereins werden von den Hausadministratoren, von 
den Sonderadministratoren, vom Vorsitzenden des Administratorenrates oder 
vom Vorstand verwaltet, abh"angig davon, ob mit ihnen Aufgaben der H"auser, 
des Sonderbereichs, des Administratorenrates oder des Vorstandes bestritten 
werden.}

\end{Artikel}

\begin{Artikel}{13}{Gewalt "uber das Rechnernetz}

\Satz{1}{Das Rechnernetz ist bis zu den Endger"aten durch Vereinseigentum
aufzubauen. Seine Funktion darf nur von vereinseigener Hardware abh"angen. 
Hardware des Studentenwerks, von beauftragten Firmen, des Leibniz-Rechenzentrums
oder anderer "offentlicher Einrichtungen sind davon ausgenommen.}

\Satz{2}{Weitergehende Rechte auf einem vereinseigenen Rechner oder
anderen Netzkomponenten werden regelm"a"sig nur Administratoren sowie
Beauftragten des Studentenwerks einger"aumt. 
"Uber begr"undete Ausnahmen beschlie"st der Administratorenrat im Einvernehmen
mit dem Vorstand.}

\Satz{3}{Der Vorsitzende des Administratorenrates f"uhrt eine Liste, wer 
auf welchem der vereinseigenen Rechner oder anderen Netzkomponenten 
weitergehende Rechte besitzt.}

\end{Artikel}

\Abschnitt{Organe}

\begin{Artikel}{14}{Organe des Vereins}

\Satz{1}{Organe des Vereins sind  
\begin{itemize}
\item die Mitgliedervollversammlung
\item die H"auser
\item der Vorstand
\item der Administratorenrat.
\end{itemize}}

\Satz{2}{Die Vertretung nach au"sen im Sinne des \S 26 BGB "uben immer zwei
Vorstandsmitglieder gemeinsam aus. Mit Wirkung gegen Dritte wird allen 
Organen untersagt, ohne Zustimmung der Mitgliedervollversammlung 
Kredite oder andere Verpflichtungen mit einer Laufzeit von mehr als zwei Jahren
aufzunehmen.}

\end{Artikel}

\UAbschnitt{Mitgliedervollversammlung}

\begin{Artikel}{15}{Aufgaben}

\Satz{1}{Die Mitgliedervollversammlung k"ummert sich um alle Fragen mit
zentraler Bedeutung f"ur den Gesamtverein.}

\Satz{2}{Sie hat insbesondere folgende Aufgaben:
\begin{itemize}
\item Genehmigung des Haushaltsplans f"ur das folgende Gesch"aftsjahr
\item Entgegennahme des Rechenschaftsberichts des Vorstands und dessen 
Entlastung
\item Wahl des Vorstands, der Sonderadministratoren und des 
Kassenpr"ufungsausschusses
\item Festsetzung der H"ohe des Mitgliedsbeitrags und der Aufnahmegeb"uhren
\item Beschlu"sfassung "uber Satzungs"anderungen und Vereinsaufl"osung
\item Beschlu"sfassung "uber die Berufung eines Mitglieds gegen einen 
Ausschlu"s durch den Vorstand
\end{itemize}}

\Satz{3}{Sie kann eine Rahmengesch"aftsordnung f"ur alle Vereinsgremien, eine
Beitragsordnung oder eine Haushaltsordnung erlassen.}

\Satz{4}{"Uber die Beschl"usse der Mitgliedervollversammlung ist ein 
Protokoll aufzunehmen, vom Versammlungsleiter zu unterzeichnen 
und vom Vorstand zu ver\-"offent\-lichen.}

\end{Artikel}

\begin{Artikel}{16}{Einberufung}

\Satz{1}{Die ordentlichen Mitgliedervollversammlungen finden zweimal 
j"ahrlich, und zwar in den Zeitr"aumen 1. - 15. Juni und 
1. - 15. Dezember, statt.}

\Satz{2}{Die Mitglieder werden mindestens zwei Wochen vor der Versammlung 
vom Vorstand durch Aushang im Vereinsanzeiger eingeladen, wobei der Tag der 
Ver\-"offent\-lichung und der Tag der Versammlung nicht mitgez"ahlt werden.}

\Satz{3}{Mit der Einladung der Mitgliedervollversammlung ist die vom 
Vorstand festgesetzte Tagesordnung mitzuteilen.}

\Satz{4}{Der Vorstand hat unverz"uglich  eine au"serordentliche 
Mitgliedervollversammlung einberufen, wenn das Vereinsinteresse es erfordert
oder wenn 15 v.H. der Mitglieder oder der Administratorenrat dies schriftlich 
unter Angabe des Zwecks und der Gr"unde fordern.}

\end{Artikel}

\UAbschnitt{H"auser} 

\begin{Artikel}{17}{Gliederung in H"auser}

\Satz{1}{Der Verein ist in H"auser untergliedert. Die H"auser regeln ihre
Belange selbst. Der Vorstand kann den H"ausern Weisungen erteilen.}

\Satz{2}{Jedem Haus wird ein Netzbereich als Hausnetz zugeordnet, von dem 
die Funktion des Gesamtnetzes nicht abh"angig sein darf.}

\Satz{3}{Die Mitglieder k"onnen die Zuordnung zu den H"ausern nicht w"ahlen.
Ausschlaggebend ist der Standort ihres Anschlusses. Im Zweifel entscheidet der 
Vorstand.}

\end{Artikel}

\begin{Artikel}{18}{Zust"andigkeit}

\Satz{1}{Die H"auser organisieren das Vereinsleben auf Hausebene.}

\Satz{2}{Sie verwalten ihr jeweiliges Hausnetz in eigener Regie.
Die technischen Standards des Administratorenrates werden dabei von
ihnen beachtet.}

\end{Artikel}

\begin{Artikel}{19}{Hausadministratoren}

\Satz{1}{Jedes Haus hat einen Hausadministrator. Pro angefangene 30 
Mitgliedern darf ein weiterer gew�hlt werden. Ausschlaggebend ist die 
Mitgliederzahl am \mbox{1. Januar} des jeweiligen Jahres.}

\Satz{2}{Die Hausadministratoren leiten ihre H"auser, verwalten die
Hausnetze und sind Mitglied im Administratorenrat.}

\Satz{3}{Die Amtsperioden der Hausadministratoren laufen vom 
\mbox{1. Juni} bis 30. November und vom \mbox{1. Dezember} bis 
\mbox{31. Mai}.}

\Satz{4}{Die Wahl der Hausadministratoren ist dem Vorstand anzuzeigen.
Dieser f"uhrt eine Liste der derzeitig t"atigen Administratoren. Mit der
Aufnahme in diese Liste ist der Gew"ahlte zum Administrator bestellt.}

\Satz{5}{W"ahlt ein Haus nicht mindestens einen Hausadministrator, oder 
sind alle Hausadministratorenstellen eines Hauses verwaist, so setzt der 
Administratorenrat einen Hausadministrator ein.}

\end{Artikel}

\begin{Artikel}{20}{Hausmitgliederversammlung}

\Satz{1}{Die Hausmitgliederversammlung regelt alle Belange ihres Hauses.}

\Satz{2}{Sie w"ahlt die Hausadministratoren, "uberwacht deren 
Amtsf"uhrung und beschlie"st "uber ihre Entlastung.}

\Satz{3}{Die Hausadministratoren legen ihr Fragen von besonderem Gewicht
zur Entscheidung vor.}

\Satz{4}{Den Vorsitz f"uhrt in der Regel einer der Hausadministratoren
des Hauses.}

\end{Artikel}

\begin{Artikel}{21}{Einberufung}

\Satz{1}{Die ordentlichen Hausmitgliederversammlungen finden zweimal 
j"ahrlich, und zwar in den Zeit\-r"aumen 15. - 31. Mai und 15. - 30. November, 
statt. Sie wird in der Regel von einem der Hausadministratoren des Hauses, im
Ausnahmefall vom Vorstand einberufen.}

\Satz{2}{Die Mitglieder werden mindestens zwei Wochen vor der Versammlung 
von den Hausadministratoren durch Aushang im Vereinsanzeiger eingeladen.}

\Satz{3}{Die Hausadministratoren k"onnen eine au"serordentliche 
Hausmitgliederversammlung einberufen. Sie m"ussen das innerhalb von drei 
Wochen tun, wenn 15 v.H. der Hausmitglieder dies fordern.}

\end{Artikel}

\UAbschnitt{Vorstand}

\begin{Artikel}{22}{Zusammmensetzung und Amtszeit}

\Satz{1}{Der Vorstand setzt sich aus dem Vorsitzenden, dem Schatzmeister, dem
Schriftf"uhrer und einem technischen Vorstand zusammen.}

\Satz{2}{Der Vorstand wird auf ein Jahr gew�hlt. Die Amtsperiode des
Vorstandes l"auft vom 1. Juli bis 30. Juni. Er bleibt so lange im Amt, bis eine
Neuwahl erfolgt. Scheidet ein Mitglied w"ahrend der Amtsperiode aus, w"ahlt der
Adminstratiorenrat f"ur die Zeit bis zur n"achsten Mitgliedervollversammlung
ein Ersatzmitglied.}

\Satz{3}{Der Vorstand haftet nicht f�r leichte Fahrl�ssigkeit.}

\end{Artikel}

\begin{Artikel}{23}{Gesch"aftsverteilung}

\Satz{1}{Der Vorsitzende k"ummert sich um alle wichtigen Belange des 
Vereins. Er koordiniert die Kontakte nach au"sen und beruft bei
Bedarf Vorstandssitzungen ein.}

\Satz{2}{Der Schatzmeister verwaltet alles Geldverm"ogen. Er ist 
verantwortlich f"ur eine ordnungsgem"a"se Buchf"uhrung. Der Schatzmeister
kann nicht gleichzeitig Administrator sein.}

\Satz{3}{Der technische Vorstand f"uhrt den Vorsitz im Administratorenrat.
Er mu"s mindestens ein Jahr lang Administrator gewesen sein.}

\Satz{4}{Der Schriftf"uhrer f"uhrt die Mitglieder- und 
Administratorenliste und sammelt und publiziert die Protokolle aller 
beschlu"sfassenden Gremien. Ferner sorgt er f"ur die Verf"ugbarkeit
aller g"ultigen Regelwerke des Vereins in der jeweils aktuellen Fassung. 
Er vertritt den Vorsitzenden, falls dieser verhindert ist.}

\end{Artikel}

\begin{Artikel}{24}{Aufgaben}

\Satz{1}{Die Vorst"ande sind die kaufm"annische und organisatorische 
F"uhrung des Vereins. Sie haben auch im Falle ihrer Abwesenheit f"ur die 
Abwicklung notwendiger Vorg"ange Sorge zu tragen.}

\Satz{2}{Die Vorst"ande sollen die Entwicklung des Vereins f"ordern und die
finanzielle Stabilit"at sicherstellen.}

\Satz{3}{Ausgaben sind vom Vorstand zu bewilligen.
Bei Ausgaben "uber 5.000 DM holt er die Genehmigung der 
Mitgliedervollversammlung ein.}

\end{Artikel}

\UAbschnitt{Administratorenrat}

\begin{Artikel}{25}{Zusammensetzung}

\Satz{1}{Mitglieder des Administratorenrates sind die Haus- und 
Sonderadministratoren sowie ein Vertreter der Hausverwaltung. 
Dieser mu"s nicht Vereinsmitglied sein.}

\Satz{2}{Jedes Ratsmitglied hat eine Stimme. Vertretung ist unzul"assig.}

\end{Artikel}

\begin{Artikel}{26}{Aufgaben}

\Satz{1}{Der Administratorenrat k"ummert sich um alle technischen Belange.}

\Satz{2}{Er legt verbindliche technische Standards fest und verwaltet
die f"ur den Betrieb des Gesamtnetzes notwendigen Netzkomponenten.}

\Satz{3}{Der Administratorenrat bestreitet seine regelm"a"sigen Aufgaben
mit einer Instandhaltungspauschale, deren H"ohe von der ordentlichen 
Mitgliedervollversammlung festgelegt wird.}

\end{Artikel}

\begin{Artikel}{27}{Vorsitz}

\Satz{1}{Der technische Vorstand ist Vorsitzender des Administratorenrats.}

\Satz{2}{Der Vorsitzende des Administratorenrats soll jederzeit auf dem 
aktuellen Stand der Netzwerkarchitektur sein. Er hat Rede- und Antragsrecht 
bei Vorstandssitzungen.}

\Satz{3}{Der Administratorenrat wird von seinem Vorsitzenden oder bei 
Bedarf von einem Vorstandsmitglied einberufen.}

\Satz{4}{Alle Administratoren stellen sicher, da"s schnell auf alle f"ur
den normalen Netzbetrieb notwendigen Komponenten zugegriffen werden kann.}

\Satz{5}{Ist der Vorsitzende des Administratorenrats l"anger als drei
Tage nicht erreichbar, so benennt er f"ur die Zeit seiner Abwesenheit
einen Stellvertreter.}

\end{Artikel}

\begin{Artikel}{28}{Sonderadministratoren}

\Satz{1}{Die Amtsperioden der Sonderadministratoren laufen vom 1. Juli bis
31. Dezember und vom 1. Januar bis 30. Juni.}

\Satz{2}{Die Sonderadministratoren werden von der Mitgliedervollversammlung
ge\-w"ahlt und durch Eintragung in die Administratorenliste durch den Vorstand
bestellt.}

\Satz{3}{Die Sonderadministratoren verf"ugen nicht "uber eigene Mittel.
Instandhaltungsaufwendungen sind aus Mitteln der Instandhaltungspauschale 
zu bestreiten und m"ussen beim Administratorenrat beantragt werden.
Neuanschaffungen bewilligt der Vorstand.}

\Satz{4}{Ist ein Sonderadministratorenamt verwaist, kann der 
Administratorenrat einen Kommissar einsetzen. Dieser hat im 
Administratorenrat keine Stimme.}

\end{Artikel}


\Abschnitt{Wenn alle Stricke rei"sen}

\begin{Artikel}{29}{Satzungs"anderungen}

\Satz{1}{Satzungs"anderungen beschlie"st die Mitgliedervollversammlung mit 
Zweidrittelmehrheit auf gemeinsamen Antrag von Vorstand und Administratorenrat
oder von 15 v.H. der Mitglieder.}

\Satz{2}{Die "Anderung ist im Vereinsregister einzutragen.}

\end{Artikel}

\begin{Artikel}{30}{Aufl"osung}

\Satz{1}{Die Aufl"osung bestimmt sich nach den Regeln der Satzungs"anderung.}

\Satz{2}{Bei Aufl"osung des Vereins f"allt das 
Verm"ogen an das Studentenwerk M"unchen, Anstalt 
des "offentlichen Rechts, mit der Ma"sgabe es unmittelbar und 
ausschlie"slich im Sinn des Vereinszwecks zu verwenden.}

\end{Artikel}

\begin{Artikel}{31}{"Ubergangsbestimmungen}

Der Verein ist durch den Vorstand zur Eintragung in das Vereinsregister 
anzumelden. Dem erstmaligen Vorsitzenden ist das Recht "ubertragen,
etwaige Satzungs"anderungen, die der Registerrichter f"ur die Eintragung
des Vereins verlangen sollte, vorzunehmen. 

\end{Artikel}

\end{document}


